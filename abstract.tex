% Abstract for Masters Thesis
% Author: Bryson Schenck
% Advisor: Josef Teichmann

\begin{abstract}
This thesis develops a convergence theory for empirical expected-signature estimators from single-path data under a segment-stationarity assumption (short path segments are shift-invariant in distribution with exponentially decaying serial dependence) and applies it to parameter calibration of two-dimensional Ornstein--Uhlenbeck processes. A segmentation and reindexing procedure is introduced that retains mean reversion and serial dependence across blocks without assuming block independence, yielding an estimator with finite-sample mean-squared-error convergence at rate $O(N^{-2/p})$ under this segment-stationarity framework combined with exponential $\alpha$-mixing. Here $N$ is a granularity parameter controlling block size and $p > 2$ denotes path regularity. Validation uses a generator-based framework that reduces expected-signature computation in linear SDEs to matrix exponentials, enabling precise numerical verification. Across parameter regimes, log--log MSE-versus-$N$ slopes between $-1.86$ and $-3.61$ are observed. For calibration, we develop a systematic hyperparameter optimization framework and compare signature-based methods against Batched MLE, a block-averaged maximum likelihood approach where parameters are estimated on multiple path segments and then averaged. In slow mean-reversion regimes, signature methods achieve 10--32\% improvement over Batched MLE, with accuracy differential scaling positively with path volatility; these gains are statistically significant across the full range (Wilcoxon signed-rank test, $p < 0.001$). Notably, in the same slow-reversion regimes, signature methods exhibit superior computational efficiency (9--15\% speedup), with per-iteration optimization cost demonstrating more favorable asymptotic scaling for high-frequency time series than Batched MLE. In such cases, GPU memory rather than computation time becomes the primary constraint on extending the method to longer paths. The results establish a quantitative convergence rate for empirical expected-signature estimation in serially dependent single-path settings and demonstrate clear advantages in both statistical accuracy and computational efficiency for calibration.
\end{abstract}